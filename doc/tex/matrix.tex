\chapter{Matrix Algebra. The {\tt matrix} library.}

To facilitate the work with matrices in C++, the {\tt matrix} library provides two classes:
\begin{itemize}
\item \verb1matrix1 for regular matrices
\item \verb1matrix_block_diag1 for block diagonal matrices
\end{itemize}
The function prototypes are defined in the header file {\tt matrix.h}.

\section{Matrix creation and manipulation}

Regular matrices are defined as objects of the \verb1matrix1 class. For example, the sentence:
\mint{cpp}|matrix a(3,4);|
creates a matrix {\tt a} with 3 rows and 4 columns. If the size is not specified,
\mint{cpp}|matrix a;|
a 1x1 matrix is created. The size of the matrix can be modified using the method {\tt dim}
\mint{cpp}|a.dim(3,4);|
or, if the total number of elements does not change, using {\tt redim}
\mint{cpp}|a.redim(1,12);|
With {\tt redim} the element values are also preserved. The number of rows and columns of a matrix object
can be retrieved using the methods {\tt nrows()} and {\tt ncols()}. For example
\begin{minted}[frame=single]{cpp}
int n,m;
matrix a(3,4);

n=a.nrows();
m=a.ncols();
\end{minted}
in this example $n=3$ and $m=4$. 

The elements of the matrix can be indexed using parenthesis. Note that, as regular C arrays, the index
of the first element is 0. There are also methods for extracting parts of the matrix. Let's see an example
\begin{minted}[frame=single]{cpp}
matrix a(3,3),row,col,block;
double elem;

a(0,0)=1;a(0,1)=2;a(0,2)=3;
a(1,0)=4;a(1,1)=5;a(1,2)=6;
a(2,0)=7;a(2,1)=8;a(2,2)=9;

elem=a(1,2); // elem=6
row=a.row(1); // Extracts the second row
col=a.col(0); // Extracts the first column
block=a.block(0,1,1,2); // Extracts the block (0-1,1-2)

\end{minted} 
After running the example, the contents of the different matrices will be
$$
\mathtt{a}=\left(
\begin{array}{ccc}
1&2&3\\4&5&6\\7&8&9
\end{array}
\right)
$$
$$
\mathtt{row}=\left(
\begin{array}{ccc}
4&5&6
\end{array}
\right)
\qquad
\mathtt{col}=\left(
\begin{array}{c}
1\\4\\7
\end{array}
\right)
\qquad
\mathtt{block}=\left(
\begin{array}{cc}
2&3\\5&6
\end{array}
\right)
$$

We can also insert parts of the matrix using the methods {\tt setrow}, {\tt setcol} and {\tt setblock}.
\begin{minted}[frame=single]{cpp}
matrix a(3,3),b;

b=ones(1,3); // Creates a 1x3 array of all ones
a.setrow(0,b);

b=ones(3,1);
a.setcol(2,b);

b=ones(2,3);
a.setblock(1,2,0,2,b);

\end{minted} 

Negative indices are interpreted starting from the end of the matrix. For example {\tt a.row(-1)} returns
the last row of the matrix {\tt a}.  

Indexing with only one parameter is also possible, being \verb|a(i,j)| equivalent to 
\verb|a(j*a.nrows()+i)|. This makes sense when working with row or column vectors, if we define
\mint{cpp}|matrix row(1,5),col(5,1);|
then \verb|row(i)| is equivalent to \verb|row(1,i)| and \verb|col(i)| is equivalent to \verb|col(i,1)|.

\pagebreak

\section{File input/output}

The method {\tt write} writes a matrix in a file, the syntax is
\mint{cpp}|write(FILE *fp, char mode)|
Here, {\tt mode} can be \verb|'t'| for text output or \verb|'b'| for binary output. Default is \verb|'t'|.
The matrix is written in column-wise order, i.e. each line represents a column of the matrix.
When called without arguments \verb|write()|, it writes the matrix in the standard output.

To read a matrix from a file we use the method \verb|read|.
\mint{cpp}|read(int nrow, int ncol, FILE *fp, char mode)|
Where we must specify the size of the matrix.

In the following example, we will write a matrix to a file and read it again.
\begin{minted}[frame=single]{cpp}
#include<stdio.h>
#include"matrix.h"
int main() {
	FILE *fp;
	matrix a(2,3);
	
	a(0,0)=1;a(0,1)=2;a(0,2)=3;
	a(1,0)=4;a(1,1)=5;a(1,2)=6;
	
	// Write the matrix to a file in binary mode
	fp=fopen("matrix.dat","wb");
	a.write(fp,'b');
	fclose(fp);
	
	// Read the matrix from file
	fp=fopen("matrix.dat","rb");
	a.read(2,3,fp,'b');
	fclose(fp);
		
	return 0;
}

\end{minted}

We can write a matrix on the screen in a more convenient format using \verb|write_fmt|. For the
previous example the sentence
\mint{cpp}|a.write_fmt("%.2f");|
will produce the following output
\begin{verbatim}
1.00 2.00 3.00
4.00 5.00 6.00
\end{verbatim}

\pagebreak

\section{Operators}

Element-wise operators for the matrix class:

\medskip

\begin{tabular}{|c|l|}
\hline
\verb|a=b| & Assignment \\
\hline
\verb|a+b| & Addition \\
\hline
\verb|a-b| & Subtraction \\
\hline
\verb|a*b| & Element-wise multiplication \\
\hline
\verb|a/b| & Element-wise division \\
\hline
\verb|a+=b| & Equivalent to \verb|a=a+b| \\
\hline
\verb|a-=b| & Equivalent to \verb|a=a-b| \\
\hline
\verb|a*=b| & Equivalent to \verb|a=a*b| \\
\hline
\verb|a/=b| & Equivalent to \verb|a=a/b| \\
\hline
\verb|+a| & Unary plus \\
\hline
\verb|-a| & Unary minus \\
\hline
\verb|a==b| & Comparison: Equal to\\
\hline
\verb|a!=b| & Comparison: Not equal to\\
\hline
\verb|a>b| & Comparison: Greater than\\
\hline
\verb|a<b| & Comparison: Less than\\
\hline
\verb|a>=b| & Comparison: Greater than or equal to\\
\hline
\verb|a<=b| & Comparison: Less than or equal to\\
\hline
\verb|a&&b| & Logical AND\\
\hline
\verb%a||b% & Logical OR\\
\hline
\end{tabular}

\medskip

The operands {\tt a} and {\tt b} can be either matrices or scalars. Element-wise operators are 
performed element by element. For example if we define
\mint{cpp}|c=a*b;|
the elements of the new matrix {\tt c} will be
\mint{cpp}|c(i,j)=a(i,j)*b(i,j)|
obviously, the two matrices must have the same size. There is one exception, when one or both of the 
dimensions are one, for example if {\tt a} is (n,m) and {\tt b} is (1,m) the matrix {\tt c} will be
(n,m) with elements
\mint{cpp}|c(i,j)=a(i,j)*b(j)|
also if {\tt a} is (n,1) and {\tt b} is (1,m), {\tt c} will be
(n,m) with 
\mint{cpp}|c(i,j)=a(i)*b(j)|

The comparison operator {\tt ==} compares two matrices element by element, so the result is a new matrix
whose elements are 1 if the corresponding elements of {\tt a} and {\tt b} are equal or 0 if they 
are different. If we want to know if two matrices are completely equal, we can use the function
\verb|isequal(a,b)| that returns 1 if {\tt a} and {\tt b} are the same and 0 otherwise.

\pagebreak

Matrix product are indicated with a comma ``\,,\,". The product of matrices {\tt a} and {\tt b} 
are expressed as:
\mint{cpp}|c=(a,b);|
The operation should be put in parentheses when necessary to avoid ambiguity. Note that the operator
``\,,\," in C has the lowest precedence, for example 
\mint{cpp}|(2*a,b+c,d)|
is equivalent to
\mint{cpp}|( (2*a) , ( (b+c) ,d ) )|

\section{Block diagonal matrices}

Another type of object included in the library are the block diagonal matrices. An object of this class
has the following structure
$$M=\left(
\begin{array}{cccc}
M_0&0&\cdots&0\\
0&M_1&\cdots&0\\
\vdots&\vdots&\ddots&\vdots\\
0&0&\cdots&M_{n-1}
\end{array}
\right)$$
where the $M_i$ are also matrices. Although the definition of a block diagonal matrix requires the blocks
$M_i$ to be square, in the current implementation they are allowed to have any size.

A block diagonal matrix is defined using the sentence
\mint{cpp}|matrix_block_diag D;|
An optional argument can be included to specify the number of blocks in the matrix (default is 1)
\mint{cpp}|matrix_block_diag D(4);|
Alternatively, the number of blocks can be changed using the sentence
\mint{cpp}|D.set_nblocks(4);|
In order to access the different blocks, we use the method \verb|block(int i)|, for exmaple
\begin{minted}{cpp}
matrix_block_diag D(3);
matrix a,b;

a=ones(2,2);
D.block(0)=a;
b=D.block(0);
\end{minted}

Individual elements can also be indexed using parentheses \verb|D(i,j)|, as with regular matrices.

A number of operators are defined in the \verb|matrix_block_diag| class:

\medskip

\begin{tabular}{|c|c|c|l|}
\hline
Operator&Operands type&Return type&\multicolumn{1}{|c|}{Description}\\
\hline
a=b&\verb|matrix_block_diag|&\verb|matrix_block_diag|&Assignment\\
\hline
a+b&\verb|matrix_block_diag|&\verb|matrix_block_diag|&Addition\\
\hline
a-b&\verb|matrix_block_diag|&\verb|matrix_block_diag|&Subtraction\\
\hline
+a&\verb|matrix_block_diag|&\verb|matrix_block_diag|&Unary plus\\
\hline
-a&\verb|matrix_block_diag|&\verb|matrix_block_diag|&Unary minus\\
\hline
\multirow{5}{*}{a*b}&\verb|matrix_block_diag|&\verb|matrix_block_diag|
&\multirow{5}{*}{Element-wise multiplication}\\
\cline{2-3}
&\verb|matrix_block_diag| \& \verb|matrix|&\verb|matrix_block_diag|&\\
\cline{2-3}
&\verb|matrix| \&  \verb|matrix_block_diag|&\verb|matrix_block_diag|&\\
\cline{2-3}
&\verb|matrix_block_diag| \& \verb|double|&\verb|matrix_block_diag|&\\
\cline{2-3}
&\verb|double| \& \verb|matrix_block_diag|&\verb|matrix_block_diag|&\\
\hline
\multirow{2}{*}{a/b}&\verb|matrix_block_diag| \& \verb|matrix|&\verb|matrix_block_diag|
&\multirow{2}{*}{Element-wise division}\\
\cline{2-3}
&\verb|matrix_block_diag| \& \verb|double|&\verb|matrix_block_diag|&\\
\hline
\multirow{3}{*}{(a,b)}&\verb|matrix_block_diag|&\verb|matrix_block_diag|
&\multirow{3}{*}{Matrix product}\\
\cline{2-3}
&\verb|matrix_block_diag| \& \verb|matrix|&\verb|matrix_block_diag|&\\
\cline{2-3}
&\verb|matrix| \&  \verb|matrix_block_diag|&\verb|matrix_block_diag|&\\
\hline
\end{tabular}

\medskip

For element-wise operators between \verb|matrix_block_diag| objects, both objects must have
exactly the same structure. Matrix product is also performed block by block, so the structure 
of the operands must be compatible.

A \verb|matrix_block_diag| object can be converted in a \verb|matrix| object using type casting.
\begin{minted}{cpp}
matrix a;
matrix_block_diag D;

a=(matrix) D;
\end{minted}


\section{Reference}

\renewcommand{\funclistcolumns}{2}
\funclist{matrix}

\subsection{A note about methods and functions}

The subroutines are divided in two types: functions and methods. Contrary to functions, methods belong
to the object and they are called using a different syntax. For example if {\tt met} is a method
of the object {\tt a} that takes one argument {\tt b} and returns a value {\tt c}, we use the sentence
\mint{cpp}|c=a.met(b)|
The same subroutine implemented as a function will be
\mint{cpp}|c=met(a,b)|
When using pointers, the dot is replaced by \verb|->|, then if \verb|p=&a| the sentence above
is equivalent to
\mint{cpp}|c=p->met(b)|
The parenthesis are needed even if the method takes no arguments, i.e. 
\verb|a.method_without_args()|.

\funcrefbegin{matrix}

\funcsec{\bf Matrix manipulation}
\subsection{Matrix manipulation}

\function{dim(n,m)}{Method}
{n (int): Number of rows\newline
m (int): Number of columns}
{Reference to current object}
{Changes the dimensions of the matrix object.}

\function{redim(n,m)}{Method}
{n (int): Number of rows\newline
m (int): Number of columns}
{Reference to current object}
{Changes the dimensions of the matrix object. The total number elements must not change. Element
values are preserved.}

\function{nrows()}{Method}{None}{int}{Returns the number of rows of the matrix.}

\function{ncols()}{Method}{None}{int}{Returns the number of columns of the matrix.}

\function{row(n)}{Method}
{n (int): Row index}
{matrix}
{Extracts row n from matrix.}

\function{col(n)}{Method}
{n (int): Column index}
{matrix}
{Extracts column n from matrix.}

\function{block(n1,n2,m1,m2)}{Method}
{n1 (int): First row index\newline
n2 (int): Last row index\newline
m1 (int): First column index\newline
m2 (int): Last column index}
{matrix}
{Extracts the block contained between the rows n1 and n2 and the columns m1 and m2.}

\function{block\_step(n1,n2,nstep,m1,m2,mstep)}{Method}
{n1 (int): First row index\newline
n2 (int): Last row index\newline
nstep (int): Row increment\newline
m1 (int): First column index\newline
m2 (int): Last column index\newline
mstep (int): Column increment}
{matrix}
{Extracts the block contained between the rows n1 and n2 and the columns m1 and m2 using increments nstep and mstep.}

\function{setrow(n,A)}{Method}
{n (int): Row index\newline
A (matrix)}
{Reference to current object}
{Inserts matrix A at row n.}

\function{setcol(n,A)}{Method}
{n (int): Column index\newline
A (matrix)}
{Reference to current object}
{Inserts matrix A at column n.}

\function{setblock(n1,n2,m1,m2,A)}{Method}
{n1 (int): First row index\newline
n2 (int): Last row index\newline
m1 (int): First column index\newline
m2 (int): Last column index\newline
A (matrix)}
{Reference to current object}
{Inserts matrix A between the rows n1 and n2 and the columns m1 and m2.}

\function{setblock\_step(n1,n2,nstep,m1,m2,mstep,A)}{Method}
{n1 (int): First row index\newline
n2 (int): Last row index\newline
nstep (int): Row increment\newline
m1 (int): First column index\newline
m2 (int): Last column index\newline
mstep (int): Column increment\newline
A (matrix)}
{Reference to current object}
{Inserts matrix A between between the rows n1 and n2 and the columns m1 and m2 
using increments nstep and mstep.}

\function{transpose()}{Method}{None}{matrix}
{Returns the tranpose of the object. Does not modify the original matrix.}

\function{fliplr()}{Method}{None}{matrix}
{Flip columns in the left-right direction. Does not modify the original matrix.}

\function{flipud()}{Method}{None}{matrix}
{Flip rows in the up-down direction. Does not modify the original matrix.}

\function{data()}{Method}{None}{Pointer to double}
{Returns a pointer to the first element in the matrix. The elements are stored consecutively in column order.}

\funcsec{\bf File input/output}
\subsection{File input/output}

\function{write(fp,mode)}{Method}
{fp (FILE *): File pointer (optional, default=stdout)\newline
mode (char): Write mode (optional, default='t')}
{int}
{Writes a matrix in the file pointed by fp in text mode (mode='t') or binary mode (mode='b').
The matrix is written in column order. Returns 0 on success, 1 otherwise.}

\function{read(n,m,fp,mode)}{Method}
{n (int): Number of rows\newline
m (int): Number of columns\newline
fp (FILE *): File pointer\newline
mode (char): Write mode (optional, default='t')}
{int}
{Reads a nxm matrix from the file pointed by fp in text mode (mode='t') or binary mode (mode='b').
The matrix is read in column order. Returns 0 on success, 1 otherwise.}

\function{write\_fmt(format,fp)}{Method}
{format (char *): Format string\newline
fp (FILE *): File pointer (optional, default=stdout)}
{None}
{Writes a matrix in the file pointed by fp using given format. The matrix is ordered such that
each line represents a row.}

\funcrefpagebreak
\funcsec{\bf Special matrices}
\subsection{Special matrices}

\function{eye(n)}{Function}
{n (int): Number of rows}
{matrix}
{Returns the nxn identity matrix.}

\function{zeros(n,m)}{Function}
{n (int): Number of rows\newline
m (int): Number of columns}
{matrix}
{Returns a nxm matrix of all zeros.}

\function{ones(n,m)}{Function}
{n (int): Number of rows\newline
m (int): Number of columns}
{matrix}
{Returns a nxm matrix of all ones.}

\function{random\_matrix(n,m)}{Function}
{n (int): Number of rows\newline
m (int): Number of columns}
{matrix}
{Returns a nxm matrix with random values between 0 and 1.}

\function{vector(x0,x1,n)}{Function}
{x0 (double): Minimum value\newline
x1 (double): Maximum value\newline
n (int): Number of elements}
{matrix}
{Returns a row vector with n equally spaced elements between x0 and x1.}

\function{vector\_t(x0,x1,n)}{Function}
{x0 (double): Minimum value\newline
x1 (double): Maximum value\newline
n (int): Number of elements}
{matrix}
{Returns a column vector with n equally spaced elements between x0 and x1.}

\funcsec{\bf Matrix functions}
\subsection{Matrix functions}

\function{max(A)}{Function}
{A (matrix)}{double}
{Returns the maximum value.}

\function{min(A)}{Function}
{A (matrix)}{double}
{Returns the minimum value.}

\function{sum(A)}{Function}
{A (matrix)}{double}
{Returns the sum of the matrix elements.}

\function{mean(A)}{Function}
{A (matrix)}{double}
{Returns the mean value of the matrix elements.}

\function{max(A,B)}{Function}
{B (matrix)\newline B (matrix)}
{matrix}
{Compares the matrices a and b and returns a new matrix C containing the larger values of each pair of 
elements C(i,j)=max(A(i,j),B(i,j)). }

\function{min(A,B)}{Function}
{A (matrix)\newline B (matrix)}
{matrix}
{Compares the matrices A and B and returns a new matrix C containing the smaller values of each pair of 
elements C(i,j)=min(A(i,j),B(i,j)). }

\function{exist(A)}{Function}
{A (matrix)}{int}
{Returns 1 if any of the elements of A is not zero, 0 otherwise. It is often used in constructions
of type \texttt{if (exist(condition))...}, where \texttt{condition} is a valid comparison. For example
\texttt{if (exist(A<0))...}.}

\function{isequal(A,B)}{Function}
{A (matrix)\newline B(matrix)}{int}
{Returns 1 if matrices A and B contain exactly the same values, 0 otherwise.}

\function{solve(b)}{Method}
{b (matrix)}{matrix}
{\texttt{x=A.solve(b)} solves the linear system $Ax=b$ and returns matrix $x$.}

\function{inv()}{Method}{None}{matrix}
{Returns the inverse of the current matrix object. The original matrix is not modified.}

\funcsec{\bf Mathematical functions}
\subsection{Mathematical functions}

\function{cos(x)}{Function}{x (matrix)}{matrix}
{Returns the cosine of x. x must be expressed in radians.}

\function{sin(x)}{Function}{x (matrix)}{matrix}
{Returns the sine of x. x must be expressed in radians.}

\function{tan(x)}{Function}{x (matrix)}{matrix}
{Returns the tangent of x. x must be expressed in radians.}

\function{acos(x)}{Function}{x (matrix)}{matrix}
{Returns the arc cosine of x in radians.}

\function{asin(x)}{Function}{x (matrix)}{matrix}
{Returns the arc sine of x in radians.}

\function{atan(x)}{Function}{x (matrix)}{matrix}
{Returns the arc tangent of x in radians.}

\function{atan2(y,x)}{Function}
{y (matrix or double)\newline
x (matrix or double)}{matrix}
{Returns the arc tangent of y/x in radians. Uses the sign of y and x to determine the quadrant.}

\function{cosh(x)}{Function}{x (matrix)}{matrix}
{Returns the hyperbolic cosine of x. x must be expressed in radians.}

\function{sinh(x)}{Function}{x (matrix)}{matrix}
{Returns the hyperbolic sine of x. x must be expressed in radians.}

\function{tanh(x)}{Function}{x (matrix)}{matrix}
{Returns the hyperbolic tangent of x. x must be expressed in radians.}

\function{exp(x)}{Function}{x (matrix)}{matrix}
{Returns $\mathrm{e}^x$.}

\function{log(x)}{Function}{x (matrix)}{matrix}
{Returns $\log(x)$.}

\function{log10(x)}{Function}{x (matrix)}{matrix}
{Returns $\log_{10}(x)$.}

\function{sqrt(x)}{Function}{x (matrix)}{matrix}
{Returns $\sqrt{x}$.}

\function{abs(x)}{Function}{x (matrix)}{matrix}
{Returns the absolute value of x.}

\function{pow(x,y)}{Function}
{x (matrix or double)\newline
y (matrix or double)}{matrix}
{Returns $x^y$.}

\function{round(x)}{Function}{x (matrix)}{matrix}
{Rounds the elements of x to the nearest integer.}

\function{floor(x)}{Function}{x (matrix)}{matrix}
{Rounds the elements of x to the nearest integer below the current value.}

\function{ceil(x)}{Function}{x (matrix)}{matrix}
{Rounds the elements of x to the nearest integer above the current value.}

\funcsec{\bf Block diagonal matrices}
\subsection{Block diagonal matrices}

\function{set\_nblocks(n)}{Method}
{n (int): Number of blocks}
{Reference to current object}
{Changes the number of blocks of the matrix\_block\_diag object.}

\function{block(n)}{Method}
{n (int): Block number}
{Reference to matrix}
{Returns a reference to the matrix in the block number n.}

\function{nblocks()}{Method}
{None}
{int}
{Returns the number of blocks.}

\function{nrows()}{Method}
{None}
{int}
{Returns the total number of rows.}

\function{ncols()}{Method}
{None}
{int}
{Returns the total number of columns.}

\function{row(n)}{Method}
{n (int): Row number}
{int}
{Extracts the row n.}

\function{transpose()}{Method}
{None}
{matrix\_block\_diag}
{Calculates the transpose.}

\function{eye(D)}{Function}
{D (matrix\_block\_diag)}
{matrix\_block\_diag}
{Returns the identity block matrix with the same structure as D.}


\funcrefend

