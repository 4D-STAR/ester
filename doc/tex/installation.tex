\chapter{Getting started}
\section{Prerequisites}

The ESTER library depends on some external libraries that should be installed in the system, namely:
\begin{itemize}
\item BLAS, CBLAS and LAPACK, for matrix algebra. There are several versions available, as 
for example:
\begin{itemize}
\item Netlib. This is the original implementation. The LAPACK library can be found at
\url{http://www.netlib.org/lapack}, and already contains BLAS, but CBLAS should be downloaded
separately from \url{http://www.netlib.org/blas}.
\item ATLAS (Automatically Tuned Linear Algebra Software). An implementation of LAPACK/BLAS that
is automatically optimized during the compilation process. It can be found at
\url{http://math-atlas.sourceforge.net/}. It contains LAPACK, BLAS and CBLAS.
\item Intel MKL. Contains an optimized version of LAPACK, BLAS and CBLAS for Intel processors.
\end{itemize}
\item PGPLOT (CPGPLOT) for graphics output (optional). It can be disabled in the 
{\tt Makefile} ({\tt make.inc}) setting the variable {\tt USE\_PGPLOT=0}.
\end{itemize}
As there are some routines written in Fortran, it is also needed to link against the standard
fortran libraries ({\tt libgfortran} for the GNU fortran compiler and {\tt libifcore} and 
{\tt libifport} for the Intel compiler).

\subsection{A note about the performance of the code}
The performance of the ESTER code depends strongly on LAPACK. To get the best results,
use an optimized (and parallelized) version.

\pagebreak

\section{Installation}
The current version of the ESTER code can be downloaded using {\tt svn} from the project server by doing
\mint{bash}|$ svn checkout http://ester-project.googlecode.com/svn/trunk/ ester| %$
or from the project website \url{http://code.google.com/p/ester-project}.

The first step is to create the file {\tt make.inc} in the directory {\tt src}
 from the two examples that are included, 
{\tt make.inc.icc} and {\tt make.inc.gcc}, for the Intel compiler and the GNU compiler respectively.
After setting the appropriate values for the compilation, we must start by doing 
\mint{bash}|ester/src$ make tables| %$
This will build some third-party libraries included in the distribution and initialise
the tables of opacity and equation of state.

We can now build the main program by doing
\mint{bash}|ester/src$ make| %$
To remove intermediate files we can also do 
\mint{bash}|ester/src$ make clean| %$
The main library is created in {\tt ester/lib/libester.so}, the header files
are in {\tt ester/include} and the executables in {\tt ester/bin}. We
can add the latter directory to the system path, for the \emph{bash}
shell 
\mint{bash}|$ export PATH="your_root_directory/ester/bin:$PATH"|
or, to make this change permanent, include this line in your {\tt .bashrc} file.

\subsection{Updating the code}
In order to update to the last version using {\tt svn}, from the root directory of the ESTER distribution
execute
\mint{bash}|ester$ svn update| %$
Depending on the update, sometimes we can do just
\mint{bash}|ester/src$ make| %$
from the {\tt src} directory. But it is safer to clean out the previous installation using
\mint{bash}|ester/src$ make distclean| %$
and then
\mint{bash}|ester/src$ make tables; make| %$

\section{Checking the installation}

To check the functionality of the program we are going to calculate the structure of a star using the default values for the parameters.
First we calculate the structure of the corresponding 1D non-rotating star. Change to your working directory and execute
\mint{bash}|$ star1d| %$
Then we use the output file (by default {\tt star.out}) as the starting point for the 2D calculation
\mint{bash}|$ star2d -i star.out -Omega_bk 0.5|   %$
This calculates the structure of a star rotating at 50$\%$ of the break-up velocity $\Omega_k=\sqrt{\frac{GM}{R_e^3}}$.

\section{Using the library}

The ESTER code can be used as a C++ library. We just have to add the following line at the beginning of our
C++ program
\mint{bash}|#include "ester.h"|
To facilitate the process of compiling and linking against the library and all its dependencies, we provide an automatically generated
script {\tt ester/utils/ester\_build} so, all you have to do is
\mint{bash}|$ ester_build your_cpp_program.cpp -o your_executable| %$






