\chapter{Basic usage}

There are three executables provided with the library:
\begin{itemize}
\item {\tt star1d} For calculating the structure of a 1D non-rotating star
\item {\tt star2d} For calculating the structure of a 2D rotating star
\item {\tt gen\_output} For generating a custom output file
\end{itemize}

\section{Configuration files}

The main configuration file for {\tt star1d} and {\tt star2d} is {\tt ester/config/star.cfg}.
This file contains the main options for the program, which are
\begin{itemize}
\item {\tt maxit} (default 200).
Maximum number of iterations. After {\tt maxit} iterations, the program
exists normally and the output file is saved, even if it has not completely converged.
\item {\tt minit} (default 1).
Minimum number of iterations. It may occur that the value of the error
for the first iteration is not representative. With this parameter we force the solver to
do at least {\tt minit} iterations. This parameter is superseded by {\tt maxit}, for example
if {\tt maxit=5} and {\tt minit=10}, the solver will do only 5 iterations.
\item {\tt tol} (default 1e-8). 
The relative tolerance for checking the convergence of the model.
\item {\tt newton\_dmax} (default 0.5).
After one step of the Newton's method, the maximum relative change
allowed for a variable is given by {\tt newton\_dmax}. If necessary the iteration is relaxed
by a parameter $h$
$$\vec x^{N+1}=\vec x^N+h \delta\vec x^N$$
according to this value.
This parameter can be used to stabilize the convergence when the initial estimation is far
from the solution.
\item {\tt output\_file} (default {\tt star.out}). Name of the output file.
\item {\tt output\_mode} (default {\tt b}). Type of the output file {\tt b} for binary
and {\tt t} for text output.
\item {\tt verbose} (default 1). Level of verbosity, from 0 (quiet) to 4.
\item {\tt plot\_device} (default {\tt /NULL}. Plotting device for PGPLOT, 
see the documentation of PGPLOT for details. For output in a X window use {\tt /XSERVE}.
To disable the graphic output use {\tt /NULL}.
\item {\tt plot\_interval} (default 10). Minimum time in seconds to update the graphic output.
\end{itemize}
All this options can be specified in the file {\tt ester/config/star.cfg} in the form
{\tt option\_name=option\_value} (one per line) and in the command line as 
{\tt -option\_name option\_value}. The options specified in the command line have precedence
over those specified in the configuration file. 

There are some additional options that can be included in the command line:
\begin{itemize}
\item[] {\tt -input\_file} {\it infile}. Use the file {\it infile} as the starting point
for the iteration.
\item[] {\tt -i} {\it infile}. Same as {\tt -input\_file} {\it infile}.
\item[] {\tt -o} {\it outfile}. Same as {\tt -output\_file} {\it outfile}.
\item[] {\tt -param\_file} {\it file}. Where {\it file} contains the parameters of the 
stellar model to be calculated (see below).
\item[] {\tt -p} {\it file}. Same as {\tt -param\_file} {\it file}.
\item[] {\tt -ascii}. Same as {\tt -output\_mode t}.
\item[] {\tt -binary}. Same as {\tt -output\_mode b}.
\item[] {\tt -noplot}. Same as {\tt -plot\_device /NULL}.
\item[] {\tt -v}{\it n}. Same as {\tt -verbose} {\it n}.
\end{itemize}

\section{Default values}

Default values to be used by {\tt star1d} or {\tt star2d} may be set up
with the files {\tt ester/config/1d\_default.par} and {\tt
ester/config/2d\_default.par}.

In the distribution of ESTER, the proposed default values are such that
the star is divided in 8 domains with 30 points in
each domains. The bounding surfaces of the domains are located at
0,0.2,0.5,0.65,0.75,0.85,0.95,0.99,1. Opacities and equation of state
are computed through OPAL tables. These inputs allow the calculation of
a 3 M$_\odot$ model (but not only of course) from scratch.

\section{{\tt star1d} input parameters}

The input parameters for {\tt star1d} can be passed in the command line
or in a text file specified with the option {\tt -param\_file} {\it file}
(or just {\tt -p} {\it file}). It can also be used simultaneously, in
this case the parameters given in the command line take precedence over
those specified in the file. In the text file they are written in the
form {\tt param\_name}={\it param\_value} and in the command line as {\tt
-param\_name} {\it param\_value}.  Here is the list of valid parameters

\begin{itemize}
\item {\tt ndomains}. The number of subdomains to use.
\item {\tt npts}. Number of points in each subdomain. It is specified as a 
comma-separated list. 
If only one value is specified, it will be used for all the subdomains, for example:
\mint{bash}|$ star1d -ndomains 4 -npts 20,20,20,20|  %$
is equivalent to
\mint{bash}|$ star1d -ndomains 4 -npts 20|  %$
\item {\tt xif}. The position of each subdomain as a comma-separated
list. The list should contain the first and the last points of the entire
domain (that should be 0 and 1), having a total of {\tt ndomains}+1
values. If only one value is specified ($\gamma$),
 the positions are calculated using the formula
$$\mathtt{xif(i)}=1-\left(1-\frac{i}{\mathtt{ndomains}}\right)^\gamma
\quad \gamma>0$$ where $\gamma=1$ corresponds to equally-spaced
subdomains, for $\gamma>1$ they are more concentrated near the surface,
and the opposite for $\gamma<1$.

\item {\tt M}. The mass in units of solar mass.
\item {\tt X}. Mass fraction of hydrogen.
\item {\tt Z}. Mass fraction of metals.
\item {\tt Xc}. Fraction of the hydrogen abundance present in the convective core. The profile
of hydrogen abundance will be in the form
$$X(\vec r)=\left\{
\begin{array}{ll}
\mathtt{X}\times \mathtt{Xc}&\mbox{if $\vec r$ is in the convective core}\\
\mathtt{X}&\mbox{otherwise}
\end{array}\right.$$
If there is no convective core, this parameter is ignored.

\item {\tt conv}. The number of subdomains within the convective core. If
{\tt conv=0} the model will be completely radiative.

\item {\tt surff}. This parameter is used for truncating the stellar model
at some point below the surface. The surface pressure will be {\tt surff}
times the "real´´ value and the boundary conditions will be adjusted
in consequence. This parameter is provided only for testing purposes as
it does not produce an accurate representation of the internal layers
of the star. For regular calculations it should be {\tt surff=1}.

\item {\tt Tc}. Initial estimation of the central temperature. To be
updated during the calculation.

\item {\tt pc}. Initial estimation of the central pressure. To be updated during the
calculation.
\item {\tt opa}. Type of opacity law. Possible values are:
\begin{itemize}
\item {\tt opal}. OPAL opacities.
\item {\tt houdek}. Houdek's interpolation of OPAL opacities (smoother).
\item {\tt kramer}. Kramer's opacity.
\end{itemize}
\item {\tt eos}. Type of equation of state. Possible values are:
\begin{itemize}
\item {\tt opal}. OPAL equation of state.
\item {\tt ideal}. Ideal gas.
\item {\tt ideal+rad}. Ideal gas with radiation.
\end{itemize}
\item {\tt nuc}. Type of nuclear reactions. At the moment, only {\tt simple} is implemented.
\item {\tt atm}. Type of atmosphere. At the moment, only {\tt simple} is implemented.
\end{itemize}
If some parameters are omitted, the program will take the value from the input file (set with
{\tt -input\_file} or {\tt -i}) or from the default parameters file in 
{\tt ester/config/1d\_default.par} when no input file is specified.

At the moment, the code does not permit to change the number of domains and/or their position
when using an input file.

\section{{\tt star2d} input parameters}

Note that the input of {\tt star2d} can be a non-rotating 1D model calculated 
with {\tt star1d}.

The program {\tt star2d} admits the same parameters than {\tt star1d}
plus some extra specific options:

\begin{itemize}
\item {\tt nth}. The number of grid points in latitude.
\item {\tt nex}. Number of radial points in the external domain.
\item {\tt Omega\_bk}. Angular velocity at the equator in units of the critical velocity
$\Omega_c=\sqrt{\frac{GM}{R_e^3}}$.
\item {\tt Ekman}. Ekman number.
\end{itemize}

\subsection{Some recipes}

The typical workflow to calculate a model starts with the calculation of
the corresponding 1D model and using it as an input for {\tt star2d}. For
example, to calculate the structure of a 5$M_\odot$ star with OPAL
opacity rotating at with $\Omega=0.7\Omega_c$ we can do:

\begin{minted}{bash}
$ star1d -M 5 -opa opal -o model1d
$ star2d -i model1d -nth 24 -Omega_bk 0.7 -o model2d
\end{minted}

As the code uses the Newton's method, sometimes it is not possible
to converge to a solution if the initial estimation is too far from
it. In this case we can use some intermediate steps.  For example, if
we want to calculate the structure of a 2.5$M_\odot$ star rotating with
$\Omega=0.9\Omega_c$, we should probably do

\medskip
\noindent\begin{tabular}{lp{4.5cm}}
\verb|$ star1d -M 2.5 -o model1d -conv 0 |  
&(Deactivate core convection to improve convergence)\\
\verb|$ star1d -i model1d -o model1d -conv 1| &	
(Re-activate core convection)\\
\verb|$ star2d -i model1d -nth 24 -Omega_bk 0.7 -o model2d| &	
(Using an intermediate value for rotation) \\
\verb|$ star2d -i model2d -nth 32 -Omega_bk 0.9 -o model2d| &	
(Calculating the final model) 
\end{tabular}
\medskip

Executing {\tt star2d} with {\tt maxit=0} can be used to interpolate a
model without recalculating it.

\medskip
\noindent
\verb|$ star2d -i model -npts |{\it npts\_new}\verb| -nth |{\it nth\_new}\verb| -o model_interp -maxit 0|

\medskip
Pressing Ctrl-C at any time during the execution of {\tt star2d} will
terminate the program, giving the possibility of finishing the current
iteration and write the result in the output file.




\section{Generating custom output files}

The output files generated by {\tt star1d} and {\tt star2d} contain just
the minimal information necessary to reconstruct the code. However,
sometimes a more detailed output is required.  This can be done using
the program {\tt gen\_output} included in the distribution. This program
reads a template from the standard input and write the result in the
standard output.  A typical call would be

\medskip
\noindent
\verb|$ gen_output |{\it model\_file}\verb| < |{\it template\_file}\verb| > |{\it output\_file}

\medskip

The template file is a regular text file with the following rules:
\begin{itemize}
\item Plain text are copied from the template to the output file. It cannot contain 
the reserved characters {\tt \$} and {\tt \textbackslash}.
\item Line breaks are ignored. To insert a line break in the output file we have to insert
a blank line in the template.
\item Variables from the model are written in the form 
\mbox{{\tt \$\{}{\it var}{\tt ,}{\it fmt}{\tt \}}}, where {\it var} is the code for the
variable (see table below) and {\it fmt} is a valid format for the C function {\it printf} 
(e.g. {\tt \%d} for an integer, {\tt \%f} for float, {\tt \%e} for exponential notation).
If {\tt fmt} is omitted {\tt\$\{}{\it var}{\tt \}} the variable is written in binary format.
\end{itemize}
\begin{longtable}{|l|p{8cm}|c|c|}
\caption{Non-exhaustive list of codes for the model variables in the template file}\\
\hline
\bf Code&\bf Description&\bf star1d&\bf star2d\\
\hline\hline
nr&\# of radial points&*&*\\
\hline
nth&\# of points in latitude&&*\\
\hline
ndomains&\# of domains&*&*\\
\hline
npts&\# of radial points in each domain&*&*\\
\hline
xif&Position of each domain&*&*\\
\hline
nex&\# of radial points in the external domain&&*\\
\hline
surff&Parameter surff (see above)&*&*\\
\hline
conv&\# of convective domains&*&*\\
\hline
Omega&Angular velocity at the equator&&*\\
\hline
Omega\_bk&Angular velocity at the equator in units of the critical velocity&&*\\
\hline
Omegac&Critical velocity $\Omega_c=\sqrt{\frac{GM}{R_e^3}}$&&*\\
\hline
X&Hydrogen abundance&*&*\\
\hline
Z&Metal abundance&*&*\\
\hline
Xc&Fraction of X at the convective core&*&*\\
\hline
rhoc&Central density&*&*\\
\hline
Tc&Central temperature&*&*\\
\hline
pc&Central pressure&*&*\\
\hline
M&Mass&*&*\\
\hline
Rp&Polar radius&*&*\\
\hline
Re&Equatorial radius&*&*\\
\hline
L&Luminosity&*&*\\
\hline
M/M\_SUN&Mass in solar units&*&*\\
\hline
Rp/R\_SUN&Polar radius in solar units&*&*\\
\hline
Re/R\_SUN&Polar radius in solar units&*&*\\
\hline
L/L\_SUN&Luminosity in solar units&*&*\\
\hline
r&Radius&*&*\\
\hline
th&Colatitude&&*\\
\hline
rex&External radius&&*\\
\hline
phi&Gravitational potential&*&*\\
\hline
phiex&Gravitational potential of the external domain&&*\\
\hline
rho&Density&*&*\\
\hline
p&Pressure&*&*\\
\hline
T&Temperature&*&*\\
\hline
w&Angular velocity&&*\\
\hline
G&Stream function for the meridional circulation&&*\\
\hline
Xr&Hydrogen abundance $X(r,\theta)$&*&*\\
\hline
N2&Squared Brunt-V\"ais\"al\"a frequency (in rd$^2$/s$^2$)&*&*\\
\hline
opa&Type of opacity&*&*\\
\hline
opa.k&Rosseland mean opacity&*&*\\
\hline
opa.xi&Thermal diffusivity ($\chi$)&*&*\\
\hline
opa.dlnxi\_lnT&$\left(\frac{\partial\log\chi}{\partial\log T}\right)_{\rho,\mu}$&*&*\\
\hline
opa.dlnxi\_lnrho&$\left(\frac{\partial\log\chi}{\partial\log\rho}\right)_{T,\mu}$&*&*\\
\hline
eos&Type of equation of state&*&*\\
\hline
eos.G1&$\Gamma_1$&*&*\\
\hline
eos.cp&$c_p$&*&*\\
\hline
eos.del\_ad&$\nabla_{ad}$&*&*\\
\hline
eos.G3\_1&$\Gamma_3-1$&*&*\\
\hline
eos.cv&$c_v$&*&*\\
\hline
eos.prad&Radiation pressure&*&*\\
\hline
eos.chi\_T&$\chi_T=\left(\frac{\partial\log p}{\partial\log T}\right)_{\rho,\mu}$&*&*\\
\hline
eos.chi\_rho&$\chi_\rho=\left(\frac{\partial\log p}{\partial\log\rho}\right)_{T,\mu}$&*&*\\
\hline
eos.d&$d=\frac{\chi_T}{\chi_\rho}=-\left(\frac{\partial\log\rho}{\partial\log T}\right)_{p,\mu}$&*&*\\
\hline
nuc.eps&Energy generation rate per unit mass&*&*\\
\hline
nuc.pp&Energy generation rate per unit mass (pp-chain)&*&*\\
\hline
nuc.cno&Energy generation rate per unit mass (CNO cycle)&*&*\\
\hline
Teff&Effective temperature at the surface $T_\mathrm{eff}(\theta)$&*&*\\
\hline
gsup&Effective gravity at the surface $g_\mathrm{eff}(\theta)$&*&*\\
\hline
D&Radial differentiation matrix $\frac{\partial}{\partial\zeta}$ for 2D models,
$\frac{\mathrm{d}}{\mathrm{d}r}$ for 1D models&*&*\\
\hline
I&Radial integration matrix&*&*\\
\hline
Dex&Radial differentiation matrix for the external domain&&*\\
\hline
Dt&Angular differentiation matrix $\frac{\partial}{\partial\theta}$ for symmetric variables&&*\\
\hline
Dtodd&Angular differentiation matrix for antisymmetric variables&&*\\
\hline
Dt2&Second order angular differentiation matrix for symmetric variables&&*\\
\hline
It&Angular integration matrix&&*\\
\hline
\end{longtable}
For 2D variables, their values at the collocation points are written in the output file
in matrix form. Each line corresponds to a different value of the colatitude $\theta$
(i.e. a different column), starting at the equator.
$$\begin{array}{cccc}
p(\zeta_0,\theta_0)&p(\zeta_1,\theta_0)&p(\zeta_2,\theta_0)&\cdots\\
p(\zeta_0,\theta_1)&p(\zeta_1,\theta_1)&p(\zeta_2,\theta_1)&\cdots\\
p(\zeta_0,\theta_2)&p(\zeta_1,\theta_2)&p(\zeta_2,\theta_2)&\cdots\\
\vdots&\vdots&\vdots&
\end{array}$$
Being $\zeta$ the radial spheroidal coordinate.
Similarly, 1D variables can be seen as a column vector and are written in one line in the
output file, terminated by a new line character.
This behavior can be inverted by writing this line in the template file
\begin{verbatim}
\conf{transpose=1}
\end{verbatim}
After this command, the variables will be written row wise, i.e. one line for each value of
the radial coordinate. Note that it does not affect variables written in binary format,
which are always column wise. To recover the original behavior we use
\begin{verbatim}
\conf{transpose=0}
\end{verbatim}

The original grid  does not contain points in the equator and the pole. If we want the values
at this points we should write
\begin{verbatim}
\conf{equator=1}
\conf{pole=1}
\end{verbatim}

By default, the output uses cgs units. If we want the normalized values used internally by
the code, we simply put
\begin{verbatim}
\conf{dim=0}
\end{verbatim}

These control commands can be written anywhere in the template file, in separated lines, 
affecting only the code that appears below them.

Let's see an example.

Template file:
\begin{minted}[frame=single]{text}
Model of ${M/M_SUN,%.2f} solar masses and R=${R,%e} cm

rotating with Omega=${Omega_bk,%f} Omegac

${nr,%d} radial points and 
${nth,%d} latitudinal points

\conf{pole=1}
\conf{equator=1}
r:

${r,%e}
Pressure:

${p,%.14e}

\end{minted} 
%$

Output file:
\begin{minted}[frame=single]{text}
Model of 2.50 solar masses and R=1.219822e+11 cm
rotating with Omega=0.900000 Omegac
240 radial points and 32 latitudinal points
r:
0.000000e+00 4.944313e+07 1.971944e+08 4.415355e+08 7.796539e+08 ...
0.000000e+00 4.944313e+07 1.971944e+08 4.415355e+08 7.796539e+08 ...
0.000000e+00 4.944313e+07 1.971944e+08 4.415354e+08 7.796533e+08 ...
0.000000e+00 4.944313e+07 1.971944e+08 4.415352e+08 7.796523e+08 ...
[...]
Pressure:
1.61049808835808e+17 1.61048890365891e+17 1.61035199104197e+17 ...
1.61049808835808e+17 1.61048890354742e+17 1.61035198927083e+17 ...
1.61049808835808e+17 1.61048890265707e+17 1.61035197512689e+17 ...
1.61049808835808e+17 1.61048890088480e+17 1.61035194697311e+17 ...
[...]
\end{minted} 


\section{Python module}

A basic python module for reading the models is included in the distribution. It is located
in {\tt ester/python/star.py}. At the moment it only works for models calculated using
{\tt star2d}. The variables in the models are defined as \emph{numpy}
arrays. Here is a little example:

\begin{minted}{python}
import sys
sys.path.append('path_to/ester/python') # include the full path to the module

from star import * # Loads the module

A=star2d('model_file')  # Loads a model 

print A.p[0,0] # Prints the central pressure

A.draw(A.w) # Makes a plot of the differential rotation
show() # Needed in non-interactive mode of matplotlib

Note that 'dotted variables' like opa.k are accessed via A.opa_k under
python.

\end{minted}





