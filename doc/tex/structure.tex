\chapter{General structure of the code}


\section{Equations to be solved}

\subsection{With dimensional variables}

We consider a lonely rotating star in a steady state. The star is
governed by the following equations for macroscopic quantities:

\begin{equation} \Delta\phi = 4\pi G\rho\end{equation}

\begin{equation} \rho T \vv\cdot\nabla s = -\Div\vF + \eps_*\end{equation}

\begin{equation} \rho \vv\cdot\na\vv = -\na P -\rho\na\phi+\vF_v\end{equation}

\begin{equation} \Div(\rho\vv) =0\end{equation}
which need to be completed by the equations of microphysics:

\greq
P\equiv P(\rho,T)\\
\khi_r \equiv \khi_r(\rho,T)\\
\eps_* \equiv \eps_*(\rho,T)\\
\egreq
and the expressions of the viscous force, which could be

\begin{equation} \vF_v = \mu(\Delta\vv + \frac{1}{3}\na\Div\vv)\end{equation}
for a compressible, constant viscosity fluid, and of the heat flux

\begin{equation} \vF = -\khi_r\na T -\frac{\khi_{\rm turb}T}{{\cal R}_M}\na s\end{equation}
where $\khi_{\rm turb}$ is the turbulent diffusion of heat and $s$ the
entropy.

This set of equations is completed by boundary conditions (discussed
below).

\subsection{Simplifications}

We simplify the system of equations by first neglecting entropy
advection by meridional circulation. We also neglect the convective
flux: thus we avoid computing stars with an outer convective envelope
where the convective flux is non-negligible. Core convection is
simplified in assuming an isentropic core. Hence, the energy/entropy
equation just reads:

\begin{equation} -\Div\vF + \eps_* = 0\end{equation}
As for the momentum equation, we split it into its azimuthal and
meridional components. As shown in \cite[][]{ELR13}, the meridional
components of the equation may be reduced to 

\begin{equation}
\rho s\Omega^2\es=\na p+\rho\na\phi
\label{baroc}
\end{equation}
where $s$ is the cylindrical radial coordinate. The vorticity equation
reduces to

\begin{equation}
\label{eq:vort_inv}
s\frac{\partial\Omega^2}{\partial z}=\ephi\cdot\frac{\na
p\times\na\rho}{\rho^2} \;.
\end{equation}
In these equation, the advection of the 'meridian momentum' has been
neglected in view of the smallness of the meridional flow.

The meridian circulation is important in the advection of angular
momentum as it balances its diffusion by viscosity. So the $\ephi$
component of the momentum equation reads:

\begin{equation}
\label{eq:angular_mom}
\na\cdot{(\rho s^2\Omega\vu)}=\na\cdot (\mu s^2\na\Omega)
\end{equation}
where $\vu$ is the meridional circulation and $\mu$ the dynamical
viscosity.


\subsection{Scaled equations}

First step is to move scaled equations with scaled quantities. We choose
to scale pressure, density and temperature by their central values and
other quantities as follows:

\begin{center}\parbox{10cm}{

Length scale $\equiv$ polar radius \dotfill $R$

Pressure scale $\equiv$ central pressure \dotfill $P_c$

Density scale $\equiv$ central density \dotfill $\rho_c$

Temperature scale $\equiv$ central temperature \dotfill $T_c$

Gravitational potential scale \dotfill  $\frac{P_c}{\rho_c}$

Angular velocity scale \dotfill $\frac{1}{R}\sqrt{\frac{P_c}{\rho_c}}$

%Time scale \dotfill $(R^2/\Phi_*)^{1/2}$

%Entropy scale \dotfill ${\cal R}_M$
}
\end{center}\bigskip

With these scalings, the equation now read:

\begin{equation} \Delta\phi = \pi_c \rho\end{equation}
where

\begin{equation} \pi_c = \frac{4\pi G\rho_c^2}{P_c}\end{equation}
Energy equation can be written

\begin{equation} \Delta T + \na\ln\chi\cdot\na T +
\Lambda\frac{\eps_*}{\chi_*} = 0\end{equation}
where

\begin{equation} \Lambda = \frac{\rho_c R^2}{T_c}\end{equation}
is a dimensional constant since $\eps_*$ and $\chi_*$ are dimensional. 

The momentum equation leads to

\begin{equation}
\rho s\Omega^2\es=\na p+\rho\na\phi
\label{barocnd}
\end{equation}
and the vorticity equation to

\begin{equation}
s\frac{\partial\Omega^2}{\partial z}=\ephi\cdot\frac{\na
p\times\na\rho}{\rho^2} \;.
\end{equation}
while angular momentum flux balance reads

\begin{equation}
\label{eq:angular_momnd}
\na\cdot{(\rho s^2\Omega\vu)}=E\na\cdot (\mu s^2\na\Omega)
\end{equation}
In this latter equation we introduced the Ekman number

\begin{equation} E = \frac{\mu_c}{\rho_c \Omega_0 R^2} \with
\Omega_0=\sqrt{\frac{P_c}{R^2\rho_c}}\; .\end{equation}
Mass conservation remains the same

\begin{equation} \Div(\rho\vu) =0\end{equation}

\subsection{Boundary conditions}

Before presenting the boundary conditions, we should define the surface
of the star: we take as its definition, the isobar where the polar
pressure is 

\beq P_s=\tau_s\frac{g_{\rm pole}}{\kappa_{\rm pole}},\eeq
On this isobar, $T=T_{\rm eff}$ only at the pole. This definition
permits a smooth continuity with the non-rotating models. This surface
will be associated with the value $\zeta=1$ of the pseudo-radial
coordinate (see the chapter on the mapping).

\begin{itemize}
\item {\bf On the gravitational potential:} regularity at the centre of the
star and vanishing at infinity. However, imposing this condition on the
surface of the star is cumbersome and leads to problem of convergence
for very flattened stars. We thus encompass the star with an empty
domain whose outer boundary is a sphere. On this outer sphere the
boundary conditions on spherical harmonics components are simply

\[ 
\dr{\phi_\ell}+\frac{(\ell+1)\phi_\ell}{r}=0
\]
which ensure the matching with a field vanishing at infinity.

\item {\bf On the velocity,} we demand stress-free conditions, namely

\[ \vv\cdot \vn =0 \quad {\rm and}\quad ([\sigma]\vn)\wedge\vn =\vzero
\]
where $[\sigma]$ is the stress tensor. However, we are in the limit of
small Ekman numbers and it is interesting (numerically) not to have to
compute this layer. However, it is necessary to take it into account
for computing the azimuthal velocity, which is otherwise undefined
\cite[e.g.][]{ELR13}. \cite{ELR13} have shown that the effect of
Ekman layer on the interior flow can be mimicked by the boundary
condition:

\begin{equation}
\label{eq:bl}
E\mu
s^2\vec{\hat\xi}\cdot\na\Omega+\psi\vec{\hat\tau}\cdot\na(s^2\Omega)=0
\qquad\mbox{on the surface}\;.
\end{equation}
where $\vec{\hat\xi}$ is a unit vector perpendicular to the surface
while $\vec{\hat\tau}$ is tangential to it. $\psi$ is the stream
function of the meridional flow. The foregoing condition is completed by

\[ \vec{\hat\xi}\cdot\vu = 0\]
at the surface.

\item {\bf The rotation speed} of the star must be specified. For this we
impose the equatorial angular velocity as a fraction of the critical
angular velocity, namely:

\[ \Omega(r=R_{\rm eq},\theta=\pi/2) =
\omega_k\sqrt{\frac{GMR^2\rho_c}{P_cR_{\rm eq}^3}}\]
Another way of specifying the angular velocity of the star is to impose
its total angular momentum (see integral constraints).

\item {\bf On temperature:} with the adopted definition of the stellar surface
and thanks to a simple model described in \cite{ELR13}, we can ascribe
to the surface of the star a temperature profile, such that

\beq T_b(\theta) = \lp\frac{g_{\rm pole}}{g_{\rm
eff}(\theta)}\frac{\kappa(\theta)}{\kappa_{\rm
pole}}\rp^{1/(n+1)}\lp\frac{-\khi_r\vn\cdot\na T}{\sigma}\rp^{1/4}\;.
\eeqn{tb}
where the polytropic index $n$ is set to $n=3$. So the temperature
boundary condition are simply:

\[ T(0) = 1 \andet T(\zeta=1,\theta)=T_b(\theta)/T_c\]


\end{itemize}



\section{Implementation}
The code is divided in several libraries. Each library implements one ore more classes designed to
handle one particular aspect of the calculation. 

\begin{itemize}
\item {\bf matrix}. Matrix algebra.
\item {\bf numdiff}. Implements Gauss-Legendre and multi-domain Gauss-Lobatto
numerical differentiation.
\item {\bf mapping}. Defines the mapping in spheroidal coordinates $r(\zeta,\theta)$. 
\item {\bf solver}. Resolution of systems of linear differential equations in 2D.
\item {\bf physics}. Calculation of physical quantities (opacity, equation of state, nuclear reaction rates).
\item {\bf star}. Provides objects and functions to calculate the structure of a star in 1D and 2D.
\item {\bf global}. Definition of global variables, e.g. physical and mathematical constants.
\item {\bf graphics}. Provides graphical output through {\tt pgplot}.
\item {\bf parser}. Parsing of configuration files and command-line arguments and file input/output.
\end{itemize} 


