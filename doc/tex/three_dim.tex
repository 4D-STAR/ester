\chapter{Stars in three dimensions}
\label{chap:3D}


\section{Introduction}

The need to deal with three-dimensional stellar models comes from the common
situation where the large scales of a star do not own any symmetry. Typically
that occurs when the star own a large-scale magnetic field with no special
symmetry of if the star  in the gravitational field of another massive object.
In the general case the situation is even more complex since the star is
usually not in a steady state: a variable tidal potential is a common case.
However, the first step to deal with theses situations is to focus on a steady
state that may only exist as a time-averaged state. But in a steady state we can
build on the previous 2D approach and devise the tools for constructing the 3D
stellar models.

The first step is to derive the 3D mapping, which maps the spheroidal shape of
the star to the spherical coordinates. We are still inspired by the work of
\cite{BGM98}, who also considered the question of 3D stellar models to deal with
binary neutron stars. Their work however aimed at including the General
Relativity formalism to include the strong gravitational field effects
\citep{BGM99}.

\section{The mapping}

We followthe 2D case described in chap. \ref{chap:mapping2D} but with a spheroid
described by

\begin{equation}
\left\{
\begin{array}{l}
r=r(\zeta,\theta',\varphi')\\
\theta=\theta'\\
\varphi=\varphi'
\end{array}
\right.
\label{themap3D}
\end{equation}
The angular variables remain the spherical ones but the radial distance now
depends on $(\theta',\varphi')$.

\subsection{Question of symmetry}

In chapter \ref{chap:mapping2D} we implicitly assumed that the star was
symmetric with respect to equator, the consequence of which  being that
the surface verifies $R(\theta)=R(\pi-\theta)$ and $A_0(\zeta)$ is a
polynomial of odd order in $\zeta$. We now discuss this choice since we
need to consider stars of any shape as long as they are topologically
equivalent to a sphere.

Let us consider a stellar surface of equation

\beq r=R(\theta,\varphi)\eeq
We can always split $R(\theta,\varphi)$ into its symmetric and
anti-symmetric parts with respect to origin, namely in the
transformation

\[ (\theta,\varphi) \tv (\pi-\theta,\varphi+\pi)\]
Hence, we write

\beq R(\theta,\varphi) = R_s(\theta,\varphi) + \delta R_a(\theta,\varphi)
\eeq
where

\greq
R_s(\theta,\varphi)=R_s(\pi-\theta,\varphi+\pi) \\
\delta R_a(\theta,\varphi) = -\delta R_a(\pi-\theta,\varphi+\pi)
\egreq
Our notation underlines the fact that deviation from central symmety is
usually small.

The splitting of $R(\theta,\varphi)$ into its symmetric and
anti-symmetric parts implies some constraints on the mapping
$r(\zeta,\theta',\varphi')$. Indeed, we require that $r\sim\zeta$ near
the centre, namely that the new coordinate system behaves just like the
spherical one. Basically, we must have

\beq \vr(r(-\zeta,\theta,\varphi),\theta,\varphi) =
\vr(r(\zeta,\pi-\theta,\varphi+\pi),\pi-\theta,\varphi+\pi) \eeq
which implies

\beq r(-\zeta,\theta,\varphi) = -r(\zeta,\pi-\theta,\varphi+\pi) \eeq

